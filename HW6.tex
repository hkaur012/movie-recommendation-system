%%%%%%%%%%%%%%%%%%%%%%%%%%%%%%%%%%%%%%%%%
% Programming/Coding Assignment
% LaTeX Template
%
% This template has been downloaded from:
% http://www.latextemplates.com
%
% Original author:
% Ted Pavlic (http://www.tedpavlic.com)
%
% Note:
% The \lipsum[#] commands throughout this template generate dummy text
% to fill the template out. These commands should all be removed when 
% writing assignment content.
%
% This template uses a Perl script as an example snippet of code, most other
% languages are also usable. Configure them in the "CODE INCLUSION 
% CONFIGURATION" section.
%
%%%%%%%%%%%%%%%%%%%%%%%%%%%%%%%%%%%%%%%%%

%----------------------------------------------------------------------------------------
%	PACKAGES AND OTHER DOCUMENT CONFIGURATIONS
%----------------------------------------------------------------------------------------



\documentclass{article}

\usepackage{fancyhdr} % Required for custom headers
\usepackage{lastpage} % Required to determine the last page for the footer
\usepackage{extramarks} % Required for headers and footers
\usepackage[usenames,dvipsnames]{color} % Required for custom colors
\usepackage{graphicx} % Required to insert images
\usepackage{listings} % Required for insertion of code
\usepackage{courier} % Required for the courier font
\usepackage{lipsum} % Used for inserting dummy 'Lorem ipsum' text into the template
\usepackage{setspace}
\usepackage{color}
\usepackage{comment}
\usepackage{caption}
\usepackage[T1]{fontenc}
\usepackage{hyperref}
\usepackage{natbib}
\usepackage{underscore}
\usepackage{subfigure}
\usepackage{fixltx2e}
\usepackage{textcomp}

\hypersetup{
    colorlinks=true,
    linkcolor=blue,
    filecolor=magenta,      
    urlcolor=cyan,
    breaklinks=true
}

\usepackage[]{algorithm2e}
\usepackage{pdfpages}
\usepackage{tikz}




%For python inclusion (http://widerin.org/blog/syntax-highlighting-for-python-scripts-in-latex-documents)
\definecolor{Code}{rgb}{0,0,0}
\definecolor{Decorators}{rgb}{0.5,0.5,0.5}
\definecolor{Numbers}{rgb}{0.5,0,0}
\definecolor{MatchingBrackets}{rgb}{0.25,0.5,0.5}
\definecolor{Keywords}{rgb}{0,0,1}
\definecolor{self}{rgb}{0,0,0}
\definecolor{Strings}{rgb}{0,0.63,0}
\definecolor{Comments}{rgb}{0,0.63,1}
\definecolor{Backquotes}{rgb}{0,0,0}
\definecolor{Classname}{rgb}{0,0,0}
\definecolor{FunctionName}{rgb}{0,0,0}
\definecolor{Operators}{rgb}{0,0,0}
\definecolor{Background}{rgb}{0.98,0.98,0.98}

% Margins
\topmargin=-0.45in
\evensidemargin=0in
\oddsidemargin=0in
\textwidth=6.5in
\textheight=9.0in
\headsep=0.25in

\linespread{1.1} % Line spacing

% Set up the header and footer
\pagestyle{fancy}
\lhead{\hmwkAuthorName} % Top left header
\chead{\hmwkClass\ (\hmwkClassInstructor\ \hmwkClassTime): \hmwkTitle} % Top center head
\chead{\hmwkClass\ (\hmwkClassInstructor): \hmwkTitle} % Top center head
\rhead{\firstxmark} % Top right header
\lfoot{\lastxmark} % Bottom left footer
\cfoot{} % Bottom center footer
\rfoot{Page\ \thepage\ of\ \protect\pageref{LastPage}} % Bottom right footer
\renewcommand\headrulewidth{0.4pt} % Size of the header rule
\renewcommand\footrulewidth{0.4pt} % Size of the footer rule

\setlength\parindent{0pt} % Removes all indentation from paragraphs

%----------------------------------------------------------------------------------------
%	CODE INCLUSION CONFIGURATION
%----------------------------------------------------------------------------------------

\definecolor{MyDarkGreen}{rgb}{0.0,0.4,0.0} % This is the color used for comments
\lstloadlanguages{Perl} % Load Perl syntax for listings, for a list of other languages supported see: ftp://ftp.tex.ac.uk/tex-archive/macros/latex/contrib/listings/listings.pdf
\lstset{language=Perl, % Use Perl in this example
        frame=single, % Single frame around code
        basicstyle=\small\ttfamily, % Use small true type font
        keywordstyle=[1]\color{Blue}\bf, % Perl functions bold and blue
        keywordstyle=[2]\color{Purple}, % Perl function arguments purple
        keywordstyle=[3]\color{Blue}\underbar, % Custom functions underlined and blue
        identifierstyle=, % Nothing special about identifiers                                         
        commentstyle=\usefont{T1}{pcr}{m}{sl}\color{MyDarkGreen}\small, % Comments small dark green courier font
        stringstyle=\color{Purple}, % Strings are purple
        showstringspaces=false, % Don't put marks in string spaces
        tabsize=5, % 5 spaces per tab
        %
        % Put standard Perl functions not included in the default language here
        morekeywords={rand},
        %
        % Put Perl function parameters here
        morekeywords=[2]{on, off, interp},
        %
        % Put user defined functions here
        morekeywords=[3]{test},
       	%
        morecomment=[l][\color{Blue}]{...}, % Line continuation (...) like blue comment
        numbers=left, % Line numbers on left
        firstnumber=1, % Line numbers start with line 1
        numberstyle=\tiny\color{Blue}, % Line numbers are blue and small
        stepnumber=5 % Line numbers go in steps of 5
}

% Creates a new command to include a perl script, the first parameter is the filename of the script (without .pl), the second parameter is the caption
\newcommand{\perlscript}[2]{
\begin{itemize}
\item[]\lstinputlisting[caption=#2,label=#1]{#1.pl}
\end{itemize}
}


%----------------------------------------------------------------------------------------
%	DOCUMENT STRUCTURE COMMANDS
%	Skip this unless you know what you're doing
%----------------------------------------------------------------------------------------

% Header and footer for when a page split occurs within a problem environment
\newcommand{\enterProblemHeader}[1]{
\nobreak\extramarks{#1}{#1 continued on next page\ldots}\nobreak
\nobreak\extramarks{#1 (continued)}{#1 continued on next page\ldots}\nobreak
}

% Header and footer for when a page split occurs between problem environments
\newcommand{\exitProblemHeader}[1]{
\nobreak\extramarks{#1 (continued)}{#1 continued on next page\ldots}\nobreak
\nobreak\extramarks{#1}{}\nobreak
}

\setcounter{secnumdepth}{0} % Removes default section numbers
\newcounter{homeworkProblemCounter} % Creates a counter to keep track of the number of problems

\newcommand{\homeworkProblemName}{}
\newenvironment{homeworkProblem}[1][Problem \arabic{homeworkProblemCounter}]{ % Makes a new environment called homeworkProblem which takes 1 argument (custom name) but the default is "Problem #"
\stepcounter{homeworkProblemCounter} % Increase counter for number of problems
\renewcommand{\homeworkProblemName}{#1} % Assign \homeworkProblemName the name of the problem
\section{\homeworkProblemName} % Make a section in the document with the custom problem count
\enterProblemHeader{\homeworkProblemName} % Header and footer within the environment
}{
\exitProblemHeader{\homeworkProblemName} % Header and footer after the environment
}

\newcommand{\problemAnswer}[1]{ % Defines the problem answer command with the content as the only argument
\noindent\framebox[\columnwidth][c]{\begin{minipage}{0.98\columnwidth}#1\end{minipage}} % Makes the box around the problem answer and puts the content inside
}

\newcommand{\homeworkSectionName}{}
\newenvironment{homeworkSection}[1]{ % New environment for sections within homework problems, takes 1 argument - the name of the section
\renewcommand{\homeworkSectionName}{#1} % Assign \homeworkSectionName to the name of the section from the environment argument
\subsection{\homeworkSectionName} % Make a subsection with the custom name of the subsection
\enterProblemHeader{\homeworkProblemName\ [\homeworkSectionName]} % Header and footer within the environment
}{
\enterProblemHeader{\homeworkProblemName} % Header and footer after the environment
}

%----------------------------------------------------------------------------------------
%	NAME AND CLASS SECTION
%----------------------------------------------------------------------------------------

\newcommand{\hmwkTitle}{Homework 6\\ Recommendation Systems } % Assignment title
%\newcommand{\hmwkDueDate}{Tuesday,\ February\ 25,\ 2020} % Due date
\newcommand{\hmwkClass}{Web Science} % Course/class
%\newcommand{\hmwkClassTime}{10:30am} % Class/lecture time
\newcommand{\hmwkClassInstructor}{Dr.Michele Weigle} % Teacher/lecturer
\newcommand{\hmwkAuthorName}{Harveen Kaur} % Your name

%----------------------------------------------------------------------------------------
%	TITLE PAGE
%----------------------------------------------------------------------------------------

\title{
\vspace{2in}
\textmd{\textbf{\hmwkClass:\ \hmwkTitle}}\\
%\normalsize\vspace{0.1in}\small{Due\ on\ \hmwkDueDate}\\
%\vspace{0.1in}\large{\textit{\hmwkClassInstructor\ \hmwkClassTime}}
\vspace{0.1in}\large{\textit{\hmwkClassInstructor}}
\vspace{3in}
}

\author{\textbf{\hmwkAuthorName}}
\date{Thursday,March 31,2020} % Insert date here if you want it to appear below your name

%----------------------------------------------------------------------------------------

\begin{document}

\maketitle
\newpage



%----------------------------------------------------------------------------------------
%	TABLE OF CONTENTS
%----------------------------------------------------------------------------------------

%\setcounter{tocdepth}{1} % Uncomment this line if you don't want subsections listed in the ToC

\newpage
\tableofcontents
\newpage

%----------------------------------------------------------------------------------------
%	PROBLEM 1
%----------------------------------------------------------------------------------------

% To have just one problem per page, simply put a \clearpage after each problem

\begin{homeworkProblem}


Find 3 users who are closest to you in terms of age, gender, and occupation.

For each of those 3 users:\\
\begin{itemize}
\item what are their top 3 (favorite) films?\\
\item what are their bottom 3 (least favorite) films?\\
\end{itemize}
Based on the movie values in those 6 tables (3 users X (favorite + least favorite)), choose a user that you feel is most like you. Feel free to note any outliers (e.g., "I mostly identify with user 123, except I did not like "Ghost" at all"). You can investigate more than just the top 3 and bottom 3 movies to find your best match.\\
This user is the substitute you.\\\\
%\problemAnswer{
\textbf{SOLUTION :}\\\\
  I followed the following steps:
  \begin{enumerate}
  
  \item\textbf{}I have assigned the variables age, gender and occupation to be 23,F and student respectively.
					
\item\textbf{}Using the given dataset, I got 3 matches.
			
\item\textbf{}The matches were user 49,159 and 477.
  	        		
							      
 \end{enumerate}
 

  
\begin{lstlisting}[language=Python, caption=Assignment6_1.py]
from operator import itemgetter
matchingUsers = []
myage = 23
myoccupation = 'student'
mygender = 'F' 
userMoviesDict = {}
userMovieRatingDict = {}
finalTopThree = {}
finalBottomThree = {}
userMovieRatingsList = []
movieRatingsList = []
matches = ''
bottomCount = 0
topCount = 0
listSize = 0

with open('u.user', 'r') as f1:
	for line in f1:
		userId,age,gender,occupation,zipcode = line.split('|')
		# if((int(age) < int(myage) and int(age) > int((myage - 3))) and (gender 
		== mygender) and (occupation == myoccupation)):
		if((int(age) == myage) and (gender 	== mygender) and (occupation
		== myoccupation)):	
			matchingUsers.append(userId)

print (matchingUsers)       	

with open('u.data', 'r') as f2:
	for line in f2:
		userId,movieId,rating,mseconds = line.split('	')
		if(userId in matchingUsers):
			if(userId in userMoviesDict):
				userMoviesDict[userId] = userMoviesDict[userId] + ":" + movieId + 
				"|" + rating 
			else :
				userMoviesDict[userId] = movieId + "|" + rating 

print('--------')
for key, value in userMoviesDict.items():
	# print(key,userMoviesDict[key])
	userMovieRatingsList = userMoviesDict[key].split(":")
	for movieRating in userMovieRatingsList:
		movie,rating = movieRating.split("|")
		userMovieRatingDict[movie] = rating
		# print(movie,rating)

	sortedRatings = sorted(userMovieRatingDict.items(), key=lambda value: value[1])
	# print("Length :",len(sortedRatings))
	bottomCount = 0
	topCount = 0
	listSize = 0
	bottomMovieData = ""
	topMovieData = ""
	for data in sortedRatings:
		listSize = listSize + 1
		if(bottomCount < 3):
			if(bottomMovieData == ""):
				bottomMovieData = str(data)
			else :
				bottomMovieData = bottomMovieData + ":" + str(data)
			bottomCount = bottomCount + 1
		if(listSize > len(sortedRatings) - 3):
			if(topMovieData == ""):
				topMovieData = str(data)
			else :
				topMovieData = topMovieData + ":" + str(data)

	finalBottomThree[key] = bottomMovieData
	finalTopThree[key] = topMovieData
	print('--------------')	
	print(finalTopThree)
	print(finalBottomThree)
	print('\n')
print (" ******* TOP FAVORITE  MOVIES ******* ")
print ("User" + "  " + "Movie Title" + "  " + "Rating")
print ("----" + "  " + "-----------" + "  " + "------")
for key, value in finalTopThree.items():	
	movieTuple = finalTopThree[key].split(":")
	for movie in movieTuple:
		movieId,rating = str(movie).split(",")
		movieId = movieId.replace("(","").replace("'","")
		with open('u.item', 'r') as file:
			for line in file:
				mid,movieTitle = line.split("|")[0:2]
				if(mid == movieId):
					print (key,"  "+ movieTitle+"  "+rating.replace(")","").
					replace("'",""))

print('\n')

print (" ******* LEAST FAVORITE MOVIES ******* ")
print ("User" + "  " + "Movie Title" + "  " + "Rating")
print ("----" + "  " + "-----------" + "  " + "------")
for key, value in finalBottomThree.items():	
	movieTuple = finalBottomThree[key].split(":")
	for movie in movieTuple:
		movieId,rating = str(movie).split(",")
		movieId = movieId.replace("(","").replace("'","")
		with open('u.item', 'r') as file:
			for line in file:
				mid,movieTitle = line.split("|")[0:2]
				if(mid == movieId):
					print (key,"  "+ movieTitle+"  "+ rating.replace(")","").
					replace("'",""))
      
\end{lstlisting}
\item\textbf The above given code generates the top favorite and least favorite movies from the selected users. 



\begin{lstlisting}

 ******* TOP FAVORITE  MOVIES *******
User  Movie Title  Rating
----  -----------  ------
49    Rosencrantz and Guildenstern Are Dead (1990)   5
49    Shallow Grave (1994)   5
49    Monty Pythons Life of Brian (1979)   5
159   Eraser (1996)   5
159   Mr. Hollands Opus (1995)   5
159   Feeling Minnesota (1996)   5
477   Nine Months (1995)   5
477   Sense and Sensibility (1995)   5
477   While You Were Sleeping (1995)   5

 ******* LEAST FAVORITE MOVIES *******
User  Movie Title  Rating
----  -----------  ------
49   Crow, The (1994)   1
49   Net, The (1995)   1
49   Ghost and the Darkness, The (1996)   1
159   Crow, The (1994)   1
159   Net, The (1995)   1
159   Ghost and the Darkness, The (1996)   1
477   Crow, The (1994)   1
477   Net, The (1995)   1
477   Ghost and the Darkness, The (1996)   1

\end{lstlisting}
\textbf{}Above given are the top favorite movies and least favorite movies.
Screenshot of the terminal is available in the repo.
%}
\end{homeworkProblem}





%----------------------------------------------------------------------------------------
%   PROBLEM 2
%----------------------------------------------------------------------------------------

\begin{homeworkProblem}
 Which 5 users are most correlated to the substitute you? Which 5 users are least correlated (i.e., negative correlation)?\\\\
%\problemAnswer{
 \textbf{SOLUTION}\\\\
 I selected user 477 as the substitute me. Then passed the preference of the substitute me to the sim_pearson function to determine the 5 most correlated users.
 
\begin{lstlisting}[language=Python, caption=Assignment6_2.py]
import csv
import math
import operator
import string
from collections import Counter
from math import sqrt

def sim_distance(prefs, p1, p2):
	'''
	Returns a distance-based similarity score for person1 and person2.
	'''

	# Get the list of shared_items
	si = {}
	for item in prefs[p1]:
		if item in prefs[p2]:
			si[item] = 1
	# If they have no ratings in common, return 0
	if len(si) == 0:
		return 0
	# Add up the squares of all the differences
	sum_of_squares = sum([pow(prefs[p1][item] - prefs[p2][item], 2) for item in
						 prefs[p1] if item in prefs[p2]])
	return 1 / (1 + sqrt(sum_of_squares))


def sim_pearson(prefs, p1, p2):
	'''
	Returns the Pearson correlation coefficient for p1 and p2.
	'''

	# Get the list of mutually rated items
	si = {}
	for item in prefs[p1]:
		if item in prefs[p2]:
			si[item] = 1
	# If they are no ratings in common, return 0
	if len(si) == 0:
		return 0
	# Sum calculations
	n = len(si)
	# Sums of all the preferences
	sum1 = sum([prefs[p1][it] for it in si])
	sum2 = sum([prefs[p2][it] for it in si])
	# Sums of the squares
	sum1Sq = sum([pow(prefs[p1][it], 2) for it in si])
	sum2Sq = sum([pow(prefs[p2][it], 2) for it in si])
	# Sum of the products
	pSum = sum([prefs[p1][it] * prefs[p2][it] for it in si])
	# Calculate r (Pearson score)
	num = pSum - sum1 * sum2 / n
	den = sqrt((sum1Sq - pow(sum1, 2) / n) * (sum2Sq - pow(sum2, 2) / n))
	if den == 0:
		return 0
	r = num / den
	return r


def topMatches(
	prefs,
	person,
	n=5,
	similarity=sim_pearson,
):
	'''
	Returns the best matches for person from the prefs dictionary. 
	Number of results and similarity function are optional params.
	'''

	scores = [(similarity(prefs, person, other), other) for other in prefs
			  if other != person]
	scores.sort()
	scores.reverse()
	return scores[0:n]


def getRecommendations(prefs, person, similarity=sim_pearson):
	'''
	Gets recommendations for a person by using a weighted average
	of every other user's rankings
	'''

	totals = {}
	simSums = {}
	for other in prefs:
	# Don't compare me to myself
		if other == person:
			continue
		sim = similarity(prefs, person, other)
		# Ignore scores of zero or lower
		if sim <= 0:
			continue
		for item in prefs[other]:
			# Only score movies I haven't seen yet
			if item not in prefs[person] or prefs[person][item] == 0:
				# Similarity * Score
				totals.setdefault(item, 0)
				# The final score is calculated by multiplying each item by the
				#   similarity and adding these products together
				totals[item] += prefs[other][item] * sim
				# Sum of similarities
				simSums.setdefault(item, 0)
				simSums[item] += sim
	# Create the normalized list
	rankings = [(total / simSums[item], item) for (item, total) in
				totals.items()]
	# Return the sorted list
	rankings.sort()
	rankings.reverse()
	return rankings


def transformPrefs(prefs):
	'''
	Transform the recommendations into a mapping where persons are described
	with interest scores for a given title e.g. {title: person} instead of
	{person: title}.
	'''

	result = {}
	for person in prefs:
		for item in prefs[person]:
			result.setdefault(item, {})
			# Flip item and person
			result[item][person] = prefs[person][item]
	return result


def calculateSimilarItems(prefs, n=10):
	'''
	Create a dictionary of items showing which other items they are
	most similar to.
	'''

	result = {}
	# Invert the preference matrix to be item-centric
	itemPrefs = transformPrefs(prefs)
	c = 0
	for item in itemPrefs:
		# Status updates for large datasets
		c += 1
		if c % 100 == 0:
			print('%d / %d' % (c, len(itemPrefs)))
		# Find the most similar items to this one
		scores = topMatches(itemPrefs, item, n=n, similarity=sim_distance)
		result[item] = scores
	return result


def getRecommendedItems(prefs, itemMatch, user):
	userRatings = prefs[user]
	scores = {}
	totalSim = {}
	# Loop over items rated by this user
	for (item, rating) in userRatings.items():
		# Loop over items similar to this one
		for (similarity, item2) in itemMatch[item]:
			# Ignore if this user has already rated this item
			if item2 in userRatings:
				continue
			# Weighted sum of rating times similarity
			scores.setdefault(item2, 0)
			scores[item2] += similarity * rating
			# Sum of all the similarities
			totalSim.setdefault(item2, 0)
			totalSim[item2] += similarity
	# Divide each total score by total weighting to get an average
	rankings = [(score / totalSim[item], item) for (item, score) in
				scores.items()]
	# Return the rankings from highest to lowest
	rankings.sort()
	rankings.reverse()
	return rankings

 

def loadMovieLens():
  # Get movie titles
	movies = {}
	for line in open('u.item'):
		(id, title) = line.split('|')[0:2]
		movies[id] = title
  # Load data
	prefs = {}
	for line in open('u.data'):
		(user, movieid, rating, ts) = line.split('\t')
		prefs.setdefault(user, {})
		prefs[user][movies[movieid]] = float(rating)
	return prefs

prefs = loadMovieLens()

with open('u.user') as tsv:
    for line in csv.reader(tsv, delimiter="|"):
        p2 = (line[0])
        p1 = '477' 
        r = sim_pearson(prefs, p1, p2) 
        with open('correlate1.csv','a') as f:
            writer=csv.writer(f)
            writer.writerow([r,p2,p1])
\end{lstlisting}
\item\textbf The above code will generate the file correlate1.csv and will give the correlation of all the users and user 477(Substitute me).

\begin{lstlisting}

 *******  NEGATIVE CORRELATION *******
User  Substitute Me  Correlation
----  -------------  -----------
677     477         -0.970725343
626     477             -1
752     477             -1
811     477             -1
856     477             -1

 ******* POSITIVE CORRELATION *******
User  Substitue ME  Correlation
----  ------------  -----------
 250    477         0.944911183
 321    477              1
 10     477              1
 67     477              1
 205    477              1
\end{lstlisting}
\textbf{}Above given is the positive and negative correlation.The data is saved in file \textbf{correlate1}.
%}
\end{homeworkProblem}
\clearpage
\newpage

%----------------------------------------------------------------------------------------
%   PROBLEM 3
%----------------------------------------------------------------------------------------

\begin{homeworkProblem}
Compute ratings for all the films that the substitute you has not seen.
\begin{itemize}
\item Provide a list of the top 5 recommendations for films that the substitute you should see.

\item Provide a list of the bottom 5 recommendations (i.e., films the substitute you is almost certain to hate).\\
\end{itemize}
%\problemAnswer{
 \textbf{SOLUTION}\\\\

I made use of the \textbf{getRecommendations} function to get the recommendations for '\textbf{Substitute Me}.
and the results of the same is saved in to a text file \textbf{recommendedMovies.txt}
The following code has been rewrited and was originally taken from  \textbf{Programming Collective Intelligence}.
 
\begin{lstlisting}[language=Python, caption=Assignment6_3.py]
import csv
import math
import operator
import string
from collections import Counter
from math import sqrt

def sim_distance(prefs, p1, p2):
	'''
	Returns a distance-based similarity score for person1 and person2.
	'''

	# Get the list of shared_items
	si = {}
	for item in prefs[p1]:
		if item in prefs[p2]:
			si[item] = 1
	# If they have no ratings in common, return 0
	if len(si) == 0:
		return 0
	# Add up the squares of all the differences
	sum_of_squares = sum([pow(prefs[p1][item] - prefs[p2][item], 2) for item in
						 prefs[p1] if item in prefs[p2]])
	return 1 / (1 + sqrt(sum_of_squares))


def sim_pearson(prefs, p1, p2):
	'''
	Returns the Pearson correlation coefficient for p1 and p2.
	'''

	# Get the list of mutually rated items
	si = {}
	for item in prefs[p1]:
		if item in prefs[p2]:
			si[item] = 1
	# If they are no ratings in common, return 0
	if len(si) == 0:
		return 0
	# Sum calculations
	n = len(si)
	# Sums of all the preferences
	sum1 = sum([prefs[p1][it] for it in si])
	sum2 = sum([prefs[p2][it] for it in si])
	# Sums of the squares
	sum1Sq = sum([pow(prefs[p1][it], 2) for it in si])
	sum2Sq = sum([pow(prefs[p2][it], 2) for it in si])
	# Sum of the products
	pSum = sum([prefs[p1][it] * prefs[p2][it] for it in si])
	# Calculate r (Pearson score)
	num = pSum - sum1 * sum2 / n
	den = sqrt((sum1Sq - pow(sum1, 2) / n) * (sum2Sq - pow(sum2, 2) / n))
	if den == 0:
		return 0
	r = num / den
	return r


def topMatches(prefs, person, n=5, similarity=sim_pearson,):
	'''
	Returns the best matches for person from the prefs dictionary. 
	Number of results and similarity function are optional params.
	'''

	scores = [(similarity(prefs, person, other), other) for other in prefs
			  if other != person]
	scores.sort()
	scores.reverse()
	return scores[0:n]


def getRecommendations(prefs, person, similarity=sim_pearson):
	'''
	Gets recommendations for a person by using a weighted average
	of every other user's rankings
	'''

	totals = {}
	simSums = {}
	for other in prefs:
	# Don't compare me to myself
		if other == person:
			continue
		sim = similarity(prefs, person, other)
		# Ignore scores of zero or lower
		if sim <= 0:
			continue
		for item in prefs[other]:
			# Only score movies I haven't seen yet
			if item not in prefs[person] or prefs[person][item] == 0:
				# Similarity * Score
				totals.setdefault(item, 0)
				# The final score is calculated by multiplying each item by the
				#   similarity and adding these products together
				totals[item] += prefs[other][item] * sim
				# Sum of similarities
				simSums.setdefault(item, 0)
				simSums[item] += sim
	# Create the normalized list
	rankings = [(total / simSums[item], item) for (item, total) in
				totals.items()]
	# Return the sorted list
	rankings.sort()
	rankings.reverse()
	return rankings


def transformPrefs(prefs):
	'''
	Transform the recommendations into a mapping where persons are described
	with interest scores for a given title e.g. {title: person} instead of
	{person: title}.
	'''

	result = {}
	for person in prefs:
		for item in prefs[person]:
			result.setdefault(item, {})
			# Flip item and person
			result[item][person] = prefs[person][item]
	return result


def calculateSimilarItems(prefs, n=10):
	'''
	Create a dictionary of items showing which other items they are
	most similar to.
	'''

	result = {}
	# Invert the preference matrix to be item-centric
	itemPrefs = transformPrefs(prefs)
	c = 0
	for item in itemPrefs:
		# Status updates for large datasets
		c += 1
		if c % 100 == 0:
			print('%d / %d' % (c, len(itemPrefs)))
		# Find the most similar items to this one
		scores = topMatches(itemPrefs, item, n=n, similarity=sim_distance)
		result[item] = scores
	return result


def getRecommendedItems(prefs, itemMatch, user):
	userRatings = prefs[user]
	scores = {}
	totalSim = {}
	# Loop over items rated by this user
	for (item, rating) in userRatings.items():
		# Loop over items similar to this one
		for (similarity, item2) in itemMatch[item]:
			# Ignore if this user has already rated this item
			if item2 in userRatings:
				continue
			# Weighted sum of rating times similarity
			scores.setdefault(item2, 0)
			scores[item2] += similarity * rating
			# Sum of all the similarities
			totalSim.setdefault(item2, 0)
			totalSim[item2] += similarity
	# Divide each total score by total weighting to get an average
	rankings = [(score / totalSim[item], item) for (item, score) in
				scores.items()]
	# Return the rankings from highest to lowest
	rankings.sort()
	rankings.reverse()
	return rankings

 

def loadMovieLens():
  # Get movie titles
	movies = {}
	for line in open('u.item'):
		(id, title) = line.split('|')[0:2]
		movies[id] = title
  # Load data
	prefs = {}
	for line in open('u.data'):
		(user, movieid, rating, ts) = line.split('\t')
		prefs.setdefault(user, {})
		prefs[user][movies[movieid]] = float(rating)
		print (prefs[user][movies[movieid]])
	return prefs

prefs = loadMovieLens()
userId = '477'
r = getRecommendations(prefs, userId) 
f = open("recommendedMovies.txt","w") 
f.write(str(r))
f.close()
\end{lstlisting}
\item\textbf The above code will generate the file recommendedMovies and will give the top 5 and bottom 5 recommendations for 477(Substitute me).\\\\
\begin{lstlisting}
***********TOP 5 RECOMMENDATIONS**************
(5.0, Wedding Gift, The (1994)) 
(5.0, Someone Elses America (1995)) 
(5.0, Some Mothers Son (1996)) 
(5.0, Santa with Muscles (1996)) 
(5.0, Saint of Fort Washington, The (1993))


***********BOTTOM 5 RECOMMENDATIONS ******************
(1.0, August (1996))
(1.0, Amityville 3-D (1983))
(1.0, American Strays (1996))
(1.0, 3 Ninjas: High Noon At Mega Mountain (1998))
(1.0, 1-900 (1994))
\end{lstlisting}
\textbf{}Above given are the top 5 and bottom 5 recommendations.
%}
\end{homeworkProblem}

%----------------------------------------------------------------------------------------
%   PROBLEM 4
%----------------------------------------------------------------------------------------

\begin{homeworkProblem}
Choose your (the real you, not the substitute you) favorite and least favorite film from the data. For each film, generate a list of the top 5 most correlated and bottom 5 least correlated films.\\

Based on your knowledge of the resulting films, do you agree with the results? In other words, do you personally like / dislike the resulting films?\\

\textbf{SOLUTION}\\
 
 Here, \textbf{transformPrefs}function was changed to get the preferences and to get the top 5 suggestions from the \textbf{topMatches} function.
The results for my favorite movie \textbf{Nine Months} and my least favorite movie of that time \textbf{Striking Distance} has been determined with positive and
negative correlations.
The following code has been rewrited and was originally taken from  \textbf{Programming Collective Intelligence}.


\begin{lstlisting}[language=Python, caption=Assignment6_1.py]
import csv
import math
import operator
import string
from collections import Counter
from math import sqrt

def sim_distance(prefs, p1, p2):
	'''
	Returns a distance-based similarity score for person1 and person2.
	'''

	# Get the list of shared_items
	si = {}
	for item in prefs[p1]:
		if item in prefs[p2]:
			si[item] = 1
	# If they have no ratings in common, return 0
	if len(si) == 0:
		return 0
	# Add up the squares of all the differences
	sum_of_squares = sum([pow(prefs[p1][item] - prefs[p2][item], 2) for item in
						 prefs[p1] if item in prefs[p2]])
	return 1 / (1 + sqrt(sum_of_squares))


def sim_pearson(prefs, p1, p2):
	'''
	Returns the Pearson correlation coefficient for p1 and p2.
	'''

	# Get the list of mutually rated items
	si = {}
	for item in prefs[p1]:
		if item in prefs[p2]:
			si[item] = 1
	# If they are no ratings in common, return 0
	if len(si) == 0:
		return 0
	# Sum calculations
	n = len(si)
	# Sums of all the preferences
	sum1 = sum([prefs[p1][it] for it in si])
	sum2 = sum([prefs[p2][it] for it in si])
	# Sums of the squares
	sum1Sq = sum([pow(prefs[p1][it], 2) for it in si])
	sum2Sq = sum([pow(prefs[p2][it], 2) for it in si])
	# Sum of the products
	pSum = sum([prefs[p1][it] * prefs[p2][it] for it in si])
	# Calculate r (Pearson score)
	num = pSum - sum1 * sum2 / n
	den = sqrt((sum1Sq - pow(sum1, 2) / n) * (sum2Sq - pow(sum2, 2) / n))
	if den == 0:
		return 0
	r = num / den
	return r


def topMatches(
	prefs,
	person,
	n=5,
	similarity=sim_pearson,
):
	'''
	Returns the best matches for person from the prefs dictionary. 
	Number of results and similarity function are optional params.
	'''

	scores = [(similarity(prefs, person, other), other) for other in prefs
			  if other != person]
	scores.sort()
	# scores.reverse()
	# return scores[0:n]
	lessfavorite = scores[:n]
	favorite = scores[-n:]
	return (lessfavorite, favorite)


def getRecommendations(prefs, person, similarity=sim_pearson):
	'''
	Gets recommendations for a person by using a weighted average
	of every other user's rankings
	'''

	totals = {}
	simSums = {}
	for other in prefs:
	# Don't compare me to myself
		if other == person:
			continue
		sim = similarity(prefs, person, other)
		# Ignore scores of zero or lower
		if sim <= 0:
			continue
		for item in prefs[other]:
			# Only score movies I haven't seen yet
			if item not in prefs[person] or prefs[person][item] == 0:
				# Similarity * Score
				totals.setdefault(item, 0)
				# The final score is calculated by multiplying each item by the
				#   similarity and adding these products together
				totals[item] += prefs[other][item] * sim
				# Sum of similarities
				simSums.setdefault(item, 0)
				simSums[item] += sim
	# Create the normalized list
	rankings = [(total / simSums[item], item) for (item, total) in
				totals.items()]
	# Return the sorted list
	rankings.sort()
	rankings.reverse()
	return rankings


def transformPrefs(prefs):
	'''
	Transform the recommendations into a mapping where persons are described
	with interest scores for a given title e.g. {title: person} instead of
	{person: title}.
	'''

	result = {}
	for person in prefs:
		for item in prefs[person]:
			result.setdefault(item, {})
			# Flip item and person
			result[item][person] = prefs[person][item]
	return result


def calculateSimilarItems(prefs, n=10):
	'''
	Create a dictionary of items showing which other items they are
	most similar to.
	'''

	result = {}
	# Invert the preference matrix to be item-centric
	itemPrefs = transformPrefs(prefs)
	c = 0
	for item in itemPrefs:
		# Status updates for large datasets
		c += 1
		if c % 100 == 0:
			print('%d / %d' % (c, len(itemPrefs)))
		# Find the most similar items to this one
		scores = topMatches(itemPrefs, item, n=n, similarity=sim_distance)
		result[item] = scores
	return result


def getRecommendedItems(prefs, itemMatch, user):
	userRatings = prefs[user]
	scores = {}
	totalSim = {}
	# Loop over items rated by this user
	for (item, rating) in userRatings.items():
		# Loop over items similar to this one
		for (similarity, item2) in itemMatch[item]:
			# Ignore if this user has already rated this item
			if item2 in userRatings:
				continue
			# Weighted sum of rating times similarity
			scores.setdefault(item2, 0)
			scores[item2] += similarity * rating
			# Sum of all the similarities
			totalSim.setdefault(item2, 0)
			totalSim[item2] += similarity
	# Divide each total score by total weighting to get an average
	rankings = [(score / totalSim[item], item) for (item, score) in
				scores.items()]
	# Return the rankings from highest to lowest
	rankings.sort()
	rankings.reverse()
	return rankings

 
def loadMovieLens():
  # Get movie titles
	movies = {}
	for line in open('u.item'):
		(id, title) = line.split('|')[0:2]
		movies[id] = title
  # Load data
	prefs = {}
	for line in open('u.data'):
		(user, movieid, rating, ts) = line.split('\t')
		prefs.setdefault(user, {})
		prefs[user][movies[movieid]] = float(rating)
	return prefs

prefs = loadMovieLens()
prefs = transformPrefs(prefs)
(less, high) = topMatches(prefs, 'Nine Months (1995)')
f = open("moviePositiveCorrelation.txt","w") 
f.write(str(less))
f.write('\n')
f.write(str(high))

(less, high) = topMatches(prefs, 'Striking Distance (1993)')
f = open("moviepNegativeCorrelation.txt","w") 
f.write(str(less))
f.write('\n')
f.write(str(high))

\end{lstlisting}
The program generates top 5 and bottom 5 recommendations for my favorite and least favorite movies and then
they are saved in to \textbf{moviePositiveCorrelation.txt} and \textbf{moviepNegativeCorrelation.txt} text files respectively.\\\\
\begin{lstlisting}

***********TOP 5 FAVORITE RECOMMENDATIONS**************
  Correlation         Movies
  -----------         ------
 1.0000000000000007  Daytrippers, The (1996) 
 1.0000000000000007  Eves Bayou (1997)
 1.0000000000000007  Free Willy 3: The Rescue (1997)
 1.0000000000000007  Garden of Finzi-Contini, The (Giardino dei Finzi-Contini, Il(1970)
 1.0000000000000013  Microcosmos: Le peuple de lherbe (1996)


***********BOTTOM 5 FAVORITE RECOMMENDATIONS******************
  Correlation         Movies
  -----------         ------
-1.0000000000000022   Steel (1997)
   -1.0               Anne Frank Remembered (1995)
   -1.0               Bloodsport 2 (1995)
   -1.0               Calendar Girl (1993)
   -1.0               Female Perversions (1996)
\end{lstlisting}

\begin{lstlisting}

***********TOP 5 LEAST FAVORITE RECOMMENDATIONS**************
  Correlation         Movies
  -----------         ------
     1.0             Screamers (1995)
     1.0             Simple Twist of Fate, A (1994)
     1.0             Sleeper (1973)
     1.0             Timecop (1994) 
1.0000000000000018   Paper, The (1994)


***********BOTTOM 5 LEAST FAVORITE RECOMMENDATIONS*****************
  Correlation         Movies
  -----------         ------
-1.0000000000000027  Bride of Frankenstein (1935)
-1.000000000000001   Boogie Nights (1997)) 
-1.000000000000001   Leaving Las Vegas (1995) 
-1.0000000000000007  Flirting With Disaster (1996)
-1.0000000000000007  Kiss the Girls (1997)
\end{lstlisting}
\textbf{}Above given are the top 5 and least 5 favorite recommendations. 

\textbf{}I chose \textbf{Nine Months} as my favorite movie. It was the only movie that I have seen. I haven't seen any of the recommended(top and least favorite) movies. Therefore, I cannot agree or disagree with the result. 

\end{homeworkProblem}



\bibliographystyle{plain}
\bibliography{A1bibFile}

\end{document}